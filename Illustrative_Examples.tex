\section{Illustrative Examples}

\textbf{Provide at least one illustrative example to demonstrate the major functions.}

To demonstrate the functionalities of the MatlbaCompat.jl library we have created \textit{\textit{\href{https://github.com/MatlabCompat/MatlabCompat.jl/blob/master/test/janus.m}{janus.m}}} - a minimalistic image analysis program in Matlab/Octave (see \textbf{supplementary video S1}). First, the program is reading a remote image - a micrograph of cultured cells, stained with a nuclear stain (A549 human lung carcinoma cells stained with Hoechst 33342 dye from a dataset used it \cite{22787215}). Next the program is using Otsu \cite{otsu1975threshold} thresholding to segment (separate) bright foreground pixels from dark background pixels. Finally, the program counts connected components in the foreground pixels, which in an idealistic case is equal to the number of cellular nuclei in the picture. \textbf{Fig. 3A} shows MATLAB\textsuperscript{TM} R2014a output upon invoking \textit{janus.m}. Code listing for \textit{janus.m} together with commentaries (separated by the symbol "\%" in Matlab/Octave) is provided bellow:\\



\verb|tic()|

\verb|% read the remote image|

\verb|img = imread('http://matlabcompat.github.io/img/example.tif');|

\verb|% compute graysacale threshold using Otsu algorithm|

\verb|threshold = graythresh(img);|

\verb|% create a binary image based on the grayscale image|

\verb|imgbw = im2bw(img, threshold);|

\verb|% display the resulting binary image|

\verb|imshow(imgbw);|

\verb|% lable each connected object in the image|

\verb|labeledbw = bwlabel(imgbw, 4);|

\verb|% count cells|

\verb|numberOfCells = max(reshape(labeledbw, 1,numel(labeledbw)));|

\verb|% display the number of objects|

\verb|disp(strcat('number of objects:', num2str(numberOfCells)));|

\verb|imshow(label2rgb(labeledbw,'jet',[0 0 0],'shuffle'));|

\verb|toc()|\\



To ensure that the same program can be understood by Julia using \href{https://github.com/MatlabCompat/MatlabCompat.jl}{MatlabCompat.jl} package either a parsing step or minimal changes to the code are necessary. These changes include basic syntax adaptation like substitution of single quotes symbol (') by double quotes ("), substitution of commentary symbol percent (\%) by hash (\#) and addition of the lines initiating packages used including at least MatlabCompat. This is basic parsing is provided by the function \textit{"rosetta()"}, like in the example below (identical to the example mentioned above):\\



\verb|using MatlabCompat|

\verb|rosetta(janus.m, janus.jl)|\\



In case of \textit{\href{https://github.com/MatlabCompat/MatlabCompat.jl/blob/master/test/janus.m}{janus.m}} the function \textit{"rosetta()"} generates immediately Julia interpretable code (see \textbf{Fig. 3B} for output in Julia 0.3.10):\\



\verb|using MatlabCompat|

\verb|importall MatlabCompat.ImageTools|

\verb|importall MatlabCompat.MathTools|

\verb|tic()|

\verb|# read the remote image|

\verb|img = imread("http://matlabcompat.github.io/img/example.tif");|

\verb|# compute graysacale threshold using Otsu algorithm|

\verb|threshold = graythresh(img);|

\verb|# create a binary image based on the grayscale image|

\verb|imgbw = im2bw(img, threshold);|

\verb|# display the resulting binary image|

\verb|imshow(imgbw);|

\verb|# lable each connected object in the image|

\verb|labeledbw = bwlabel(imgbw, 4);|

\verb|# count cells|

\verb|numberOfCells = max(reshape(labeledbw, 1,numel(labeledbw)));|

\verb|disp(strcat("number of objects:", num2str(numberOfCells)));|

\verb|# display the number of objects|

\verb|imshow(label2rgb(labeledbw,"jet",[1 1 1],"shuffle"));|

\verb|toc()|\\



Arguably, one of the most attractive features Julia may provide for the research software is the potential for performance optimization. To challenge the results \href{http://julialang.org/benchmarks/}{published} on the official website of the language we have executed the published standard algorithms performance tests in slightly different environment for a new version of Julia (0.3.10) and two different versions of MATLAB\textsuperscript{TM} (R2014a and R2015a). The tests include Fibonacci recursion (fib), parse integers (parse\_int), Mandelbrot set: complex arithmetic and comprehensions (mandel), numeric vector sorting (quicksort), slow $\pi$ series (pi\_sum), random matrix statistics (rand\_mat\_stat), large random number generation and multiplication of random matrices (rand\_mat\_mul).

The test environment in the published results was: Intel® Xeon® CPU E7-8850 2.00GHz CPU with 1TB of 1067MHz DDR3 RAM, running Linux in single core execution. In our case the test environment was Intel® Core\textsuperscript{TM} i7-4790K CPU E7-8850 4.00GHz (8 CPUs) CPU with 16GB of 1600MHz DDR3 RAM, running Windows 8.1 Pro 64-bit (6.3, Build 9600) without restriction to a single core. Results (\textbf{Tab. 1}) suggest small differences in execution time. The differences are most pronounced for Julia's most efficient Fibonacci Recursion (fib). This can be attributed to the difference in test settings and the version of the language. Despite the small differences, we were able to faithfully reproduce the pattern and the orders of magnitudes of the published results. Surprisingly MATLAB\textsuperscript{TM} R2015a performed slightly slower than R2014a in our test environment.

\begin{table} 
    \begin{tabular}{cccc}
    \hline
        \textbf{Algorithm} & \textbf{Reported 2014a\textsuperscript{1}} & \textbf{Tested  2014a\textsuperscript{2}} & \textbf{Tested 2015a\textsuperscript{2}}\\
        \hline
        fib & 1989.78 & 1032.79 & 1101.73 \\ 
        \hline
        parse\_int & 971.90 & 350.81 & 408.64 \\ 
        \hline
        mandel & 46.17 & 47.69 & 47.62 \\ 
        \hline
        quicksort & 69.07 & 47.28 & 57.80 \\ 
        \hline
        pi\_sum & 1.28 & 1.09 & 1.10 \\ 
        \hline
        rand\_mat\_stat & 5.64 & 1.41 & 3.91 \\ 
        \hline
        rand\_mat\_mul & 1.03 & 2.84 & 2.86 \\
        \hline
        \multicolumn{4}{l}{\textsuperscript{1}MATLAB\textsuperscript{TM} Folds Julia 0.3.7} \\
        \multicolumn{4}{l}{\textsuperscript{2}MATLAB\textsuperscript{TM} Folds Julia 0.3.10} \\
    \end{tabular}
    \caption{Julia to MATLAB Performance Tests Comparison}
\end{table}\\
  
We have next compared the execution times of our model image analysis program identical to \textit{"janus.m"} and \textit{"janus.jl"}. Median results of 1000 iterations in the test environment mentioned above suggested that MATLAB\textsuperscript{TM} R2014a performed 0.29 folds of Julia 3.10 and R2015a 0.30 folds of Julia 3.10. We concluded that performance of Julia and MATLAB\textsuperscript{TM} was comparable for this specific case. This may be caused by the fact that MATLAB\textsuperscript{TM} is highly optimized for vector operations used in this program.