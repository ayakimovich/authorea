\section{Illustrative Examples}
\label{} 

\textbf{Provide at least one illustrative example to demonstrate the major functions.}

To demonstrate the functionalities of the MatlbaCompat.jl library we have created \href{https://github.com/MatlabCompat/MatlabCompat.jl/blob/dev/test/janus.m}{janus.m} - a minimalistic image analysis program in Matlab/Octave. First, the program is reading a remote image - a micrograph of cultured cells, stained with a nuclear stain (A549 human lung carcinoma cells stained with Hoechst 33342 dye from a dataset used it \cite{22787215}). Next the program is using Otsu \cite{otsu1975threshold} thresholding to segment (separate) bright foreground pixels from dark background pixels. Finally, the program counts connected components in the foreground pixels, which in an idealistic case is equal to the number of cellular nuclei in the picture. Code listing for janus.m together with commentaries (separted by the symbol "\%" in Matlab/Octave) is provided bellow:

\verb|tic()|

\verb|img = imread('http://matlabcompat.github.io/img/example.tif'); % read the remote image|

\verb|threshold = graythresh(img); % compute graysacale threshold using Otsu algorithm|

\verb|imgbw = im2bw(img, threshold); % create a binary image based on the grayscale image|

\verb|imshow(imgbw); % display the resulting binary image|

\verb|labeledbw = bwlabel(imgbw, 4); % lable each connected object in the image|

\verb|numberOfCells = max(reshape(labeledbw, 1,numel(labeledbw))); % count cells|

\verb|disp(strcat('number of objects:', num2str(numberOfCells)));% display the number of objects|

\verb|imshow(label2rgb(labeledbw))|

\verb|toc()|

To ensure that the same program can be understood by Julia using MatlabCompat.jl package either a parsing step or minimal changes to the code are necessary. These changes include basic syntax adaptation like substitution of single quotes symbol (') by double quotes ("), substitution of commentary symbol percent (\%) by hash (#) and addition of the lines initiating packages used including at least MatlabCompat. This is basic parsing is provided by the function \textit{"rossetta()"}, like in the example below (identical to the example mentioned above):

\verb|using MatlabCompat|

\verb|rossetta(janus.m, janus.jl)|

In case of \href{https://github.com/MatlabCompat/MatlabCompat.jl/blob/dev/test/janus.m}{janus.m} the function \textit{"rossetta()"} generates immediately Julia interpretable code:

\verb|using MatlabCompat|

\verb|using MatlabCompat.ImageTools|

\verb|using MatlabCompat.MathTools|

\verb|tic()|

\verb|img = imread("http://matlabcompat.github.io/img/example.tif"); # read the remote image|

\verb|threshold = graythresh(img); # compute graysacale threshold using Otsu algorithm|

\verb|imgbw = im2bw(img, threshold); # create a binary image based on the grayscale image|

\verb|imshow(imgbw); # display the resulting binary image|

\verb|labledbw = bwlabel(imgbw, 4); # lable each connected object in the image|

\verb|numberOfCells = max(reshape(labledbw, 1,numel(labledbw))); # count cells|

\verb|disp(strcat("number of objects:", num2str(numberOfCells)));|

\verb|imshow(label2rgb(labledbw))# display the number of objects|

\verb|toc()|



\begin{table} 
    \begin{tabular}{ c c c c }
         & Reported MATLAB 2014a (Folds Julia 0.3.7) & MATLAB 2014a (Folds Julia 0.3.10) & MATLAB 2015a (Folds Julia 0.3.10) \\ 
        fib & 1989.78 & 1032.79 & 1101.73 \\ 
        parse\_int & 971.90 & 350.81 & 408.64 \\ 
        mandel & 46.17 & 47.69 & 47.62 \\ 
        quicksort & 69.07 & 47.28 & 57.80 \\ 
        pi\_sum & 1.28 & 1.09 & 1.10 \\ 
        rand\_mat\_stat & 5.64 & 1.41 & 3.91 \\ 
        rand\_mat\_mul & 1.03 & 2.84 & 2.86 \\ 
    \end{tabular} 
    \caption{Table 1. Julia to MATLAB Performance Tests Comparison} 
\end{table}

  
  
  
  
  
  
  
  
  
  
  
  
  
  
  
  
  
  
  