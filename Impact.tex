\section{Impact}

Beyond image analysis illustrated on simple examples here, research code written in Matlab/Octave is used in many scientific and engineering disciplines. Multiple scripts written and used in research groups across the world serve to complete a broad range of research tasks from data acquisition to analysis and visualization. Porting the Matlab/Octave code to Julia using \href{https://github.com/MatlabCompat/MatlabCompat.jl}{MatlabCompat.jl} may significantly broaden the applicability of their code. While specific new research questions are subject to research context and algorithm design, features of Julia language can significantly improve research process and the exchange of tools and datasets. These features include flexible Perl-like regular expression for text parsing, R-like dataframes for working with large scientific, Python-like (IPython) browser based integrated development environment (IJulia package), ability to turn prototype code into a web-applications, easy-to-use web based cloud computing environment provided by \href{https://www.juliabox.org/}{Juliabox} and many others. Performance optimization of the code combined with using the cloud solutions for Julia for a non-computational research group can mean budget and infrastructure needs could differ between few laptops and a full scale computer cluster needed to accomplish a research project. This, in turn, can have profound implications for feasibility of a large number of research projects.

\href{https://github.com/MatlabCompat/MatlabCompat.jl}{MatlabCompat.jl} has been included in Julia's official packages and is widely discussed across the community. We hope attract more code developers to the crowd-based \href{https://github.com/MatlabCompat/MatlabCompat.jl}{MatlabCompat.jl} development. For this purpose the development of the package is based on \href{http://matlabcompat.github.io/}{Github organization} rather than an individual account. By this we hope to establish a community board and maximize the involvement into the project.
  
  