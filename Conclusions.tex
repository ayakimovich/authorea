\section{Conclusions}

\textbf{Set out the conclusion of this original software publication.}

We have developed \href{https://github.com/MatlabCompat/MatlabCompat.jl}{MatlabCompat.jl} - a Julia package aimed to assist conversion of Matlab/Octave code to Julia providing one-way compatibility between these languages. Being a library package, \href{https://github.com/MatlabCompat/MatlabCompat.jl}{MatlabCompat.jl} code is by definition reusable and ment to be taken up by the community. We welcome contributions from the community following the \href{http://matlabcompat.github.io/contribute.html}{contribution guide}. Quality of our software is continuously monitored by test programs that exercise all the main features of the software in a CI environment. Furthermore, all the code written using \href{https://github.com/MatlabCompat/MatlabCompat.jl}{MatlabCompat.jl} can easily be portable to other platforms including mobile through \href{https://www.juliabox.org/}{JuliaBox} web service. We show practical applicability of the software on a simplistic Matlab/Octave image analysis script ("janus.m"). This scripts is readily executable in MATLAB\textsuperscript{TM} R2014A and, upon simple parsing by a \href{https://github.com/MatlabCompat/MatlabCompat.jl}{MatlabCompat.jl} function (\textit{rosetta()}) converting it to "janus.jl", thi script is readily executable in Julia 0.3.10.

We were able to reproduce performance comparison tests published on the \href{http://julialang.org/benchmarks/}{official website of Julia} within the order of magnitude of execution time, suggesting a much higher performance of Julia implementation of Fibonacci recursion over MATLAB\textsuperscript{TM}. While our test environment was different, we argue that it was closer to the "out-of-the-box" user experience on a modern desktop computer. Furthermore, we believe it was important to reconfirm the performance test results in a slightly different (potentially less optimized) environment to show the stability and reproducibility of those results.

The three orders of magnitude faster performance of Julia compared to MATLAB\textsuperscript{TM} can be explained by the fact that Fibonacci recursion is a great representation of algorithms programmed using nested loops. However, performance of our test use case program was slightly lower in Julia, than in MATLAB\textsuperscript{TM}. This can be explained by the fact that MATLAB\textsuperscript{TM} is highly optimized for vectorized tasks. These test results suggest a word of caution: not every type of code converted from Matlab/Octave will be faster in Julia compared to MATLAB\textsuperscript{TM}. However, this might change once it will be made possible to easily generate binaries from the source code - i.e. static compilation.

Altogether, our results suggest that converting Matlab/Octave code to Julia may significantly improve performance of research software requiring large amount of iterations. However, this is a subject to algorithms used, software implementation, input data and other details. \href{https://github.com/MatlabCompat/MatlabCompat.jl}{MatlabCompat.jl} can assist in accomplishing such a conversion effort. We invite the community to help further develop \href{https://github.com/MatlabCompat/MatlabCompat.jl}{MatlabCompat.jl}, which can ultimately improve the functionality and performance of the research software.