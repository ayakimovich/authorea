Figure 3: \textbf{Step-by-step Output of the Model Use Case Program "janus"}. (A) - Output in MATLABTM R2014A. (B) - Output in Julia 0.3.10. (C) - Illustrative example of how simple code in MATLAB/Octave and it's translated version in Julia can be used to perform automatic quantification of a dataset published in \cite{Moy_2009}, where the authors performed an \textit{in vivo} high-throughput screen for conditions preventing bacterial infection related death of \textit{Caenorhabditis elegans} model organisms. Based on their published dataset, worms placed in 384 microtiter plates, infected with equal amounts of \textit{Enterococcus faecalis}. Respective wells were treated with either ampicillin (treated, positive control for infection related death inhibition) or DMSO (non-treated, mock control) or a compound of unknown efficacy (not shown here). Each well was imaged in transmission light (TL, unspecific wide-field image) or SYTOX Orange fluorescence (specific for dead worms) \cite{Moy_2009}, representative micrographs of which are shown. (D) - Here we have quantified from the published raw micrographs of SYTOX Orange fluorescence micrographs for either treated or non-treated control wells using either "janus2.m" or automatically translated "janus2.jl" microprograms with identical results. Death signal on the graph is measured as number of SYTOX Orange positive pixels min/max normalized between 0 and 7000. Error bars correspond to the standard deviation from three different experimental replica wells micrographs for each condition provided in the published dataset \cite{Moy_2009}.