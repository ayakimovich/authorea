\section{Motivation and significance}
\label{} 

\textbf{Introduce the scientific background and the motivation for developing the software.}
Matlab/Octave is a technical computing programming language widely used through academic community. MATLAB has been introduced in late 1970s as an open source programming language for technical computing and later commercialized by MathWorks Inc. Being product of Mathworks Inc., MATLAB became popular among primarily engineers providing a flexible solutions for linear algebra, differential equations and a number of other problems engineers deal with on a daily basis \cite{moore2014matlab}. Itroduced in the end of 1980s its syntaxial sibling, GNU Octave, offered an open source alternative to the commercial product of Mathworks \cite{eaton1997gnu}.

Yet more than a tool for engineers, in the last three decades MATLAB/Octave has become a \textit{de facto} default tool for developing scientific code, often used to solve problems it was not initially designed for. This, in turn, lead to an accumulation of prototype-like scientific software solutions unable to scale up to the promise and, thus, limiting the research work. To address this, in recent years a number of new programming languages and frameworks has been proposed including SciPy for Python \cite{jones2001open, Olivier_2002}, SCILAB \cite{Campbell_2009} and others. While some projects managed to successfully undertake the effort of rewriting their research software into one of these new frameworks \cite{17076895, 21349861}, for many this represents a major problem due to programming language differences.

A newly introduced open source dynamic high level technical programing language Julia aims at shattering the paradigm that high level prototype code has to be inherently inefficient \cite{bezanson2012julia, bezanson2014julia}.  Unlike other high level technical computing languages Julia offers abstract types, multiple dispatch of methods allowing reusing them dynamically, and other attractive features of a modern programming language. The source code written in Julia is executed using just-in-time (JIT) compilation based on its low level virtual machine (LLVM) may deliver performance matching performance of the iconic C language \cite{bezanson2012julia, bezanson2014julia}. Much like \textit{Esperanto} of technical computing the syntax of Julia combines to a larger extend syntaxes of MATLAB/Octave, Python, R, Perl and many other languages. However as close as it may seem to MATLAB/Octave the code written in MATLB/Octave requires significant modifications to be understood by Julia.

\textbf{Explain why the software is important, and describe the exact (scientific) problem(s) it solves.}
Here we present 


\textbf{Indicate in what way the software has contributed (or how it will contribute in the future) to the process of scientific discovery; if available, this is to be supported by citing a research paper using the software.}


\textbf{Provide a description of the experimental setting (how does the user use the software?).}


\textbf{Introduce related work in literature (cite or list algorithms used, other software etc.).}

    
    
    
    
    
    
    
    
    
    
    
    
    
    
    
    