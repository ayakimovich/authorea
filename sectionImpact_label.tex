\section{Impact}
\label{Impact} 

\textbf{Indicate in what way new research questions can be pursued as a result of the software (if any).}
\textbf{Indicate in what way, and to what extent, the pursuit of existing research questions is improved (if so).}

Beyond image analysis illustrated on a simple example here research, code written in Matlab/Octave is used in many scientific and engineering disciplines. Multiple scripts written and used in research groups across the world serve to complete a broad range of research tasks from data acquisition to analysis and visualization.

More than just substituting the Matlab/Octave interpreter used by researchers in their daily analytical work, porting the Matlab/Octave code to Julia using MatlabCompat.jl may significantly broaden the applicability of their code. While specific new research questions are subject to research context and algorithm design, features of Julia language can significantly improve research process and the exchange of tools and datasets. These features include flexible Perl-like regular expression for text parsing, R-like dataframes for working with large scientific , Python-like (iPython) browser based integrated development environment (iJulia package), ability to turn prototype code into a web-applications, easy-to-use web based cloud computing environment provided by Juliabox and many others.

\textbf{Indicate in what way the software has changed the daily practice of its users (if so).}


\textbf{Indicate how widespread the use of the software is within and outside the intended user group.}


\textbf{Indicate in what way the software is used in commercial settings and/or how it led to the creation of spin-off companies (if so).}

  