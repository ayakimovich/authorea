\section{Impact}

\textbf{Indicate in what way new research questions can be pursued as a result of the software (if any).}
\textbf{Indicate in what way, and to what extent, the pursuit of existing research questions is improved (if so).}

Beyond image analysis illustrated on a simple example here, research code written in Matlab/Octave is used in many scientific and engineering disciplines. Multiple scripts written and used in research groups across the world serve to complete a broad range of research tasks from data acquisition to analysis and visualization. More than just substituting the Matlab/Octave interpreter used by researchers in their daily analytical work, porting the Matlab/Octave code to Julia using MatlabCompat.jl may significantly broaden the applicability of their code. While specific new research questions are subject to research context and algorithm design, features of Julia language can significantly improve research process and the exchange of tools and datasets. These features include flexible Perl-like regular expression for text parsing, R-like dataframes for working with large scientific, Python-like (IPython) browser based integrated development environment (IJulia package), ability to turn prototype code into a web-applications, easy-to-use web based cloud computing environment provided by Juliabox and many others.

\textbf{Indicate in what way the software has changed the daily practice of its users (if so).}

Once legacy Matlab/Octave code is converted to Julia using MatlabCompat.jl this can change the entire way researcher is using the code, updating it and sharing it. Instead of keeping the code on the local lab computers using Julia researchers are strongly encouraged and provided with the necessary means to use version control systems (Git) and share the code immediately from the development environment (Github, IJulia, \href{https://www.juliabox.org/}{Juliabox}). Performance optimization of the code combined with using the cloud solutions for Julia for a non-computational research group can mean budget and infrastructure needs could differ between few laptops and a full scale computer cluster needed to accomplish a research project. This, in turn, can have profound implications for feasibility of a large number of research projects.

\textbf{Indicate how widespread the use of the software is within and outside the intended user group.}

MatlabCompat.jl has been included in Julia's official packages and is widely discussed across the community. We hope attract more code developers to the crowd-based MatlabCompat.jl development. For this purpose the development of the package is based on \href{http://matlabcompat.github.io/}{Github organization} rather than an individual account. By this we hope to establish a community board and maximize the involvement into the project. Furthermore, the whole Github community is kindly invited to fork MatlabCompat.jl repository, add new features and create pull requests to merge this features back into MatlabCompat.jl, details are available in the \href{http://matlabcompat.github.io/contribute.html}{contribution guide}.

\textbf{Indicate in what way the software is used in commercial settings and/or how it led to the creation of spin-off companies (if so).}

  
  
  
  
  