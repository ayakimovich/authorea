\section{Software description}
\label{Description} 

\textbf{Describe the software in as much as is necessary to establish a vocabulary needed to explain its impact. }

MatlabCompat.jl is a Julia language package that helps converting your code in Matlab/Octave into Julia. Much like Julia language itself MatlabCompat.jl is fully Github based and a work in progress inviting everyone to collaborate. It is set up as a Github organization, which is open for new members and the library source code is open for Github based pull requests. The longterm goal of the library is to allow effortless compatibility of Matlab/Octave research code to Julia.

A typical code conversion strategies for a high level programming languages involve either translation/adaptation of the syntax of the source language to the syntax of the destination (object) language \cite{Ledley_1962} or usage of the library translating the code of the source language "as is" in the environment of the destination language (\textbf{Fig. 1}). While first strategy is always available for a developer, MatlabCompat.jl makes the second strategies possible yet not exclusive.

More often than not, certain code adaptation may become necessary or desired for several reasons. First, being constantly under dynamic development both MatlabCompat.jl and Julia language itself might either not cover the necessary functionality of Matlab/Octave or this functionality might altered/limited for syntaxial and other reasons. Second, pursuing exclusively strategy 2 developers risk creating non-optimal \textit{"stab"} Julia code. To avoid that MatlabCompat.jl aims at providing minimal Julia-invasive compatibility of Matlab/Octave code, where developers get fully exposed to Julia being able to utilize all of its features. Therefore developers are suggested to pursue hybrid strategy requiring code refactoring to make their Matlab/Octave work in Julia yet minimizing the effort needed to achieve it.

\subsection{Software Architecture}
\label{Architecture} 

\textbf{Give a short overview of the overall software architecture; provide a pictorial component overview or similar (if possible). If necessary provide implementation details.}

Following the minimal Julia invasivity concept mentioned, MatlabCompat.jl has a well-structured architecture allowing minimization of namespace conflicts with Julia's \textit{Base} package (\textbf{Fig. 2}). A typical usage of a Julia package involves invoking \textit{"using"} command:

\verb|using MatlabCompat|

The namespace conflicts between Julia and Matlab/Octave are partially inevitable given the syntaxial closeness of Julia and Matlab/Octave. E.g. function \textit{max()} in Matlab/Octave is closer to function \textit{maximum()} in Julia, while \textit{max()} function name is taken in Julia. To address this type of issues Julia language avoids global declarations and uses local scopes. Typically for other Julia packages invoking using with a package would export all the functions of the package into the scope. Yet in case of MatlabCompat.jl to avoid namespace conflicts with \textit{"Base"} package, invoking the line above for MatlabCompat package would export into the scope only those functions names of which are not conflicting with \textit{"Base"} package. To substitute the respective function from the \textit{"Base"} package by a function from MatlabCompat.jl package wit the same name (for Matlab/Octave compatibility reasons) a user would have to explicitly export the respective module of MatlabCompat. For example to use max() function from MatlabCompat.jl package rather than from \textit{"Base"} package:

\verb|using MatlabCompat, MatlabCompat.MathTools|

\verb|max([1 2 3])|

\verb|3|


MatlabCompat.jl follows a rather simple modules organization convention. Namely, all the modules are name with function purposes based self-explanatory names in \textit{"CamelCase"} with capitalized first letter. Modules may contained submodules if necessary. All functions modules contain should be either aliases (wrappers) or reimplementations of Matlab/Octave functions named the same way. The only module representing an exeption in this case is module \textit{"Support"}, which contains functions, not having an analogue neither in Matlab/Octave nor in Julia languages, necessary for other functions in the package.

\subsection{Software Functionalities}
\label{Functionalities} 

\textbf{Present the major functionalities of the software.}

Currently the major functionalities of the MatlabCompat.jl include syntax aliases or reimplementations of Matlab/Octave functions organized in modules with self-explanatory naming: \textit{"StringTools"}, \textit{"MathTools"}, \textit{"ImageTools"}, \textit{"Support"} (see \textbf{Fig. 2} and \href{http://matlabcompat.github.io/help.html}{help}). Like the Julia language, MatlabCompat.jl is in an active development and is actively receiving new contributions, including new modules. For an up-to-date overview of functions provided by different modules of the package please see the \href{http://matlabcompat.github.io/help.html}{help}. For pragmatic reasons the Matlab/Octave to Julia compatibility functions covered by MatlabCompat.jl at the current stage are strongly shifted towards image analysis and processing functions, including e.g. \textit{graythresh()} for Otsu thresholding \cite{otsu1975threshold}, \textit{bwlabel()} for labeling and subsequent counting of e.g. bright objects over dark background (discussed in detail below) and other functions.

\subsection{Sample code snippets analysis (optional)}
\label{Snippets}

An example of functionality provided by the MatlabCompat.jl package standing out, but worth mentioning is covered by function \textit{rossetta()} (\textit{"Support" module}). This function is aimed to be used as the entry point for Matlab/Octave-to-Julia conversion of your code. It is aimed to parse m-files (first argument) making basic syntax changes required for the code compatibility to Julia and then save the jl-file (second argument) to the specified path:

\verb|using MatlabCompat|

\verb|rossetta(janus.m, janus.jl)|

You can test the functionality of this code using the minimalistic \href{https://github.com/MatlabCompat/MatlabCompat.jl/blob/dev/test/janus.m}{janus.jl} program discussed below.
  
  
  
  
  